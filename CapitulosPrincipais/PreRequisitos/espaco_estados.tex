\section{Representação do Espaço de Estados no Scilab}
Considera o seguinte sistema:

$$
\begin{bmatrix}
\dot{x}_{1}\\\dot{x}_{2}
\end{bmatrix} = \begin{bmatrix}
0 & 1\\
-2 & -3
\end{bmatrix} \begin{bmatrix}
x_{1}\\x_{2}
\end{bmatrix}+\begin{bmatrix}
1 & 0\\
0 & 1
\end{bmatrix} \begin{bmatrix}
u_{1}\\u_{2}
\end{bmatrix}
$$

$$
y=\begin{bmatrix}
1 & 0
\end{bmatrix} \begin{bmatrix}
x_{1} \\ x_{2}
\end{bmatrix} + \begin{bmatrix}
0 & 0
\end{bmatrix} \begin{bmatrix}
u_{1} \\ u_{2}
\end{bmatrix}
$$

Sabe-se que a representação geral de equações de estado é:

$$\textbf{$\dot{x}$} = \textbf{$Ax$} + \textbf{$Bu$}$$
$$\textbf{$y$} = \textbf{$Cx$} + \textbf{$Du$}$$

Para representar esse sistema no Scilab basta representar as matrizes A, B, C e D e utilizar a função \verb|syslin|. Essa função nos permite gerar um sistema linear a partir das matrizes de representação no espaço de estados\footnote{Para saber mais ver \verb|https://help.scilab.org/docs/6.0.1/pt\_BR/syslin.html|}:

 \begin{verbatim}
     --> A = [0 1;-2 -3];
     --> B = [1 0;0 1];
     --> C = [1 0];
     --> D = [0 0];
     --> [sistema] = syslin('c', A, B, C, D);
 \end{verbatim}
\textbf{Obs}: O primeiro parâmetro \verb|'c'| significa que estamos trabalhando no domínio do tempo contínuo. Para trabalhar com tempo discreto use \verb|'d'|.

Ao chamar a matriz \verb|sistema| no console do Scilab, obtemos:

\begin{verbatim}
sistema  = 


 sistema(1)  (state-space system:)

  "lss"  "A"  "B"  "C"  "D"  "X0"  "dt"

 sistema(2)= A matrix =

   0.   1.
  -2.  -3.

 sistema(3)= B matrix =

   1.   0.
   0.   1.

 sistema(4)= C matrix =

   1.   0.

 sistema(5)= D matrix =

   0.   0.

 sistema(6)= X0 (initial state) =

   0.
   0.

 sistema(7)= Time domain =

  "c"

\end{verbatim}