\section{Conversão de Função de Transferência para Espaço de Estados}
 Considere a seguinte a seguinte função de tranferência:
 
 $$H(s) = \frac{2s+1}{s^{2}+3s+1}$$
 
 Para obter as matrizes de representação no espaço de estado utiliza-se a função \verb|tf2ss|\footnote{Para mais informações ver \verb|https://help.scilab.org/docs/5.3.3/en\_US/tf2ss.html|}. Então, depois de definir $H(s)$, basta aplicá-lo na função.
 
 \begin{verbatim}
    --> H_ss = tf2ss(H);
 \end{verbatim}
 
 Chamando \verb|H_ss| obtêm-se:
 
 \begin{verbatim}
     H_ss  = 


 H_ss(1)  (state-space system:)

  "lss"  "A"  "B"  "C"  "D"  "X0"  "dt"

 H_ss(2)= A matrix =

  -2.4   0.2
   2.2  -0.6

 H_ss(3)= B matrix =

  -1.5491933
   0.7745967

 H_ss(4)= C matrix =

  -1.2909944  -2.220D-16

 H_ss(5)= D matrix =

   0.

 H_ss(6)= X0 (initial state) =

   0.
   0.

 H_ss(7)= Time domain =

    []
 \end{verbatim}
 
 Onde é observa-se que:
 
 $$
 A=\begin{bmatrix} 
    -2.4 & 0.2\\
     2.2 & -0.6 
 \end{bmatrix}
 $$
 $$
 B=\begin{bmatrix} 
    -1.5491933\\0.7745967
 \end{bmatrix}
 $$
 $$
 C=\begin{bmatrix} 
    -1.2909944 & -2.220D-16
 \end{bmatrix}
 $$
 $$
 D=[0]
 $$
 
 \section{Conversão de Espaço de Estados para Função de Transferência}
 Considera a seguinte representação em espaço de estados.
 
 $$
\begin{bmatrix}
\dot{x}_{1}\\\dot{x}_{2}
\end{bmatrix} = \begin{bmatrix}
0 & 1\\
-2 & -3
\end{bmatrix} \begin{bmatrix}
x_{1}\\x_{2}
\end{bmatrix}+\begin{bmatrix}
1 & 0\\
0 & 1
\end{bmatrix} \begin{bmatrix}
u_{1}\\u_{2}
\end{bmatrix}
$$

$$
y=\begin{bmatrix}
1 & 0
\end{bmatrix} \begin{bmatrix}
x_{1} \\ x_{2}
\end{bmatrix} + \begin{bmatrix}
0 & 0
\end{bmatrix} \begin{bmatrix}
u_{1} \\ u_{2}
\end{bmatrix}
$$

Agora isola-se as matrizes A, B, C e D:

$$A=\begin{bmatrix}
    0 & 1 \\
    -2 & -3
\end{bmatrix}$$

$$B=\begin{bmatrix}
    1 & 0 \\
    0 & 1
\end{bmatrix}$$

$$C=\begin{bmatrix}
    1 & 0
\end{bmatrix}$$

$$D=\begin{bmatrix}
    0 & 0
\end{bmatrix}$$

Agora basta usar a função \verb|syslin|, para criar um sistema linear a partir dessas matrizes e depois utilizar a função \verb|ss2tf|\footnote{Para saber mais ver \verb|https://help.scilab.org/docs/5.3.3/en\_US/ss2tf.html|}.

\begin{verbatim}
    --> [sistema] = syslin('c', A, B, C, D);
    --> sistema_tf = ss2tf(sistema);
\end{verbatim}

Chamando \verb|sistema_tf| obtêm-se:

\begin{verbatim}
    sistema_tf  = 

                         
         3 +s         1      
       ---------  ---------  
       2 +3s +s^2  2 +3s +s^2
\end{verbatim}  

\newpage

Ou seja:

$$\frac{Y(s)}{U_{1}(s)} = \frac{s+3}{s^{2}+3s+2}$$
\\
e
\\
$$\frac{Y(s)}{U_2{s}} = \frac{1}{s^{2}+3s+2}$$
